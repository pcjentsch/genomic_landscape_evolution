\documentclass{article}

\usepackage{amsmath}
\usepackage{amssymb}
\usepackage{tikz}
\usepackage{tikzit}
\usepackage{cite}
%% -------------------------------------- Declare the layers
\pgfdeclarelayer{nodelayer}
\pgfdeclarelayer{edgelayer}
\pgfsetlayers{edgelayer,nodelayer,main}
% Node styles
\tikzstyle{black circle}=[fill=white, draw=black, shape=circle, tikzit shape=circle]
\tikzstyle{Arrow}=[->]

% Edge styles
\tikzstyle{new edge style 0}=[->]

\begin{document}


\title{Topic 2}
\maketitle
Compare epidemiological parameters, including patterns of viral spread between health regions and hotspots/non-hotspots within each province (phylodynamic studies with simple compartmental epidemic models linked to viral genealogies).

\begin{itemize}
    \item Fit model to populations in Ontario + Quebec, use estimates of epidemiological parameters to conclude something about effects of different NPIs on transmission
    \item Can extend to include spatial spread, other heterogeneity if the data is sufficient
\end{itemize}


\section{Model}

Compartmental differential equation models have been used extensively to model competing viral strains (see e.g. \cite{Alizon_van_Baalen_2008, van_Baalen_Sabelis_1995, Lipsitch_Colijn_Cohen_Hanage_Fraser_2009, Nicoli_Ayabina_Trotter_Turner_Colijn_2015}). We can further subdivide population into high-SES vs low-SES. This raises some further questions that are difficult to answer. E.g. What is the extent of transmission between high and low SES groups? 

Another approach that seems to be used more widely with Sars-CoV-2 be a semi-mechanistic, renewal equation based approach as in \cite{Cauchemez_Nouvellet,Fraser_2007,Mishra_Berah_Mellan_Unwin_Vollmer_Parag_Gandy_Flaxman_Bhatt_2020,Nouvellet_Cori, Wallinga_Lipsitch_2007}.

Some assumptions: 
\begin{itemize}
    \item Distinguish infections as VOC and non-VOC (possibly more detail?)
    \item VOC and non-VOC have different infection rates, recovery times
    \item Co-infection negligible 
    % \item Separate susceptible population by SES
    \item Time scale short enough that recovery from either infection grants immunity
    \item Vaccination immunity does not wane over model timescale ?
    \item Immunity from vaccination commutes with immunity from infection 
    \item Vaccination reduces all infection parameters by the same proportion ?
    \item Vaccination rate is constant (not necessary since we have data on exact vaccination rates)
\end{itemize}




\begin{figure}[h!]
    \centering
    \begin{tikzpicture}
	\begin{pgfonlayer}{nodelayer}
		\node [style=black circle] (0) at (0, 0) {};
		\node [style=black circle] (1) at (-1, 0) {};
		\node [style=black circle] (2) at (1, 0) {};
		\node [style=black circle] (3) at (2, 0) {};
		\node [style=black circle] (4) at (-2, 0) {};
		\node [style=rectangle] (5) at (-2, 1) {$V_i$};
		\node [style=rectangle] (6) at (-2, 4) {$S_i$};
		\node [style=rectangle] (7) at (-2, 2) {$R_i$};
		\node [style=rectangle] (8) at (-2, 3) {I_i};
	\end{pgfonlayer}
\end{tikzpicture}

\caption{State diagram of model with vaccination and explicit competition}
\end{figure}

\begin{table}[h!]
    \begin{center}
    \begin{tabular}{c|r|c}
        symbol  & description & source \\
        \hline
        \hline
        $\beta_N$ & Transmission rate (Non-VoC) & \\
        $\sigma_N$ & Inverse of latent period (Non-VoC) & \\
        $\gamma_N$ & Recovery rate (Non-VoC) & \\
        $\beta_V$ &  Transmission rate (VoC) & \\
        $\sigma_V$ & Inverse of latent period (VoC) & \\
        $\gamma_V$ & Recovery rate (VoC)  & \\
        $\eta$ & Vaccination rate (time dependent?) & \\
        $C(t)$ & Time varying contact rate due to lockdown protocol & \\
        \hline
        $S$ & Susceptible & \\
        $I^V$ & Infected with VoC & \\
        $I^N$ & Infected with Non-VoC & \\
        $I^{VN}$ & Recovered from Non-VoC, infected with VoC & \\
        $I^{NV}$ & Recovered from VoC, infected with non-VoC & \\
        $E^{VN}$ & Recovered from Non-VoC, exposed to VoC & \\
        $E^{NV}$ & Recovered from VoC, exposed to non-VoC & \\
        $R$ & Recovered & \\
        $E^V$ & Exposed to VoC & \\
        $E^N$ & Exposed to Non-VoC & \\
        $S^V$ & Recovered from Non-VoC, susceptible to VoC  & \\
        $S^N$ & Recovered from VoC, susceptible to non-VoC  & \\
        \hline
        $S_V$ & Vaccinated, Susceptible & \\
        $I^V_V$ & Vaccinated, Infected with VoC & \\
        $I^N_V$ & Vaccinated, Infected with Non-VoC & \\
        $I^{VN}_V$ & Vaccinated, Recovered from Non-VoC, infected with VoC & \\
        $I^{NV}_V$ & Vaccinated, Recovered from VoC, infected with non-VoC & \\
        $E^{VN}_V$ & Vaccinated, Recovered from Non-VoC, exposed to VoC & \\
        $E^{NV}_V$ & Vaccinated, Recovered from VoC, exposed to non-VoC & \\
        $R_V$ & Vaccinated, Recovered & \\
        $E^V_V$ & Vaccinated, Exposed to VoC & \\
        $E^N_V$ & Vaccinated, Exposed to Non-VoC & \\
        $S^V_V$ & Vaccinated, Recovered from Non-VoC, susceptible to VoC  & \\
        $S^N_V$ & Vaccinated, Recovered from VoC, susceptible to non-VoC  & \\
    \end{tabular}
    \caption{Table of symbols}
    \end{center}
\end{table}


\newpage
\clearpage
\section{Useful Data}
\subsection{General}
\begin{itemize}
    \item RePositive
    \item Investigation\_Lineage
    \item Investigation\_Mutation
    \item Accurate\_Episode\_Date
    \item Case\_Reported\_Date
    \item Client\_Postal\_Code
    \item Client\_Province
    \item Client\_Address\_City
    \item Age\_At\_Time\_of\_Illness
    \item Likely\_Acquisition
    \item Outbreak\_Number
    \item Setting\_Combined
    \item Epidemiologic\_Linkage
    \item Epidemiologic\_Link\_Status
\end{itemize}


\subsection{Social determinants of health}

\begin{itemize}
    \item Res\_AdultDevServices
    \item Res\_AdultYouthAddiction
    \item Res\_ChildrensRes\_Site
    \item Res\_CorrectionalF
    \item Res\_HomelessShelter
    \item Res\_LTCH
    \item LTCH\_Resident
    \item LTCH\_HCW
    \item Res\_RetirementHome
    \item Res\_SupportiveHousing
    \item Res\_OtherCongregateCare
    \item Res\_VAWorAHT\_Site
    \item Ses\_Income
    \item SES\_HHSize\_Num
    \item Occ\_LTCH
    \item Ses\_Race\_Black
    \item Ses\_Race\_East\_Southeast\_Asian
    \item Ses\_Race\_South\_Asian
    \item Ses\_Race\_White
    \item Ses\_Race\_Middle\_Eastern
    \item Ses\_Race\_Latino
    \item Immunocompromised
\end{itemize}

\bibliographystyle{plain}
\bibliography{ref.bib}

\end{document}